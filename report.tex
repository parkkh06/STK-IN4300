% Options for packages loaded elsewhere
\PassOptionsToPackage{unicode}{hyperref}
\PassOptionsToPackage{hyphens}{url}
%
\documentclass[
]{article}
\usepackage{amsmath,amssymb}
\usepackage{lmodern}
\usepackage{iftex}
\ifPDFTeX
  \usepackage[T1]{fontenc}
  \usepackage[utf8]{inputenc}
  \usepackage{textcomp} % provide euro and other symbols
\else % if luatex or xetex
  \usepackage{unicode-math}
  \defaultfontfeatures{Scale=MatchLowercase}
  \defaultfontfeatures[\rmfamily]{Ligatures=TeX,Scale=1}
\fi
% Use upquote if available, for straight quotes in verbatim environments
\IfFileExists{upquote.sty}{\usepackage{upquote}}{}
\IfFileExists{microtype.sty}{% use microtype if available
  \usepackage[]{microtype}
  \UseMicrotypeSet[protrusion]{basicmath} % disable protrusion for tt fonts
}{}
\makeatletter
\@ifundefined{KOMAClassName}{% if non-KOMA class
  \IfFileExists{parskip.sty}{%
    \usepackage{parskip}
  }{% else
    \setlength{\parindent}{0pt}
    \setlength{\parskip}{6pt plus 2pt minus 1pt}}
}{% if KOMA class
  \KOMAoptions{parskip=half}}
\makeatother
\usepackage{xcolor}
\usepackage[margin=1in]{geometry}
\usepackage{graphicx}
\makeatletter
\def\maxwidth{\ifdim\Gin@nat@width>\linewidth\linewidth\else\Gin@nat@width\fi}
\def\maxheight{\ifdim\Gin@nat@height>\textheight\textheight\else\Gin@nat@height\fi}
\makeatother
% Scale images if necessary, so that they will not overflow the page
% margins by default, and it is still possible to overwrite the defaults
% using explicit options in \includegraphics[width, height, ...]{}
\setkeys{Gin}{width=\maxwidth,height=\maxheight,keepaspectratio}
% Set default figure placement to htbp
\makeatletter
\def\fps@figure{htbp}
\makeatother
\setlength{\emergencystretch}{3em} % prevent overfull lines
\providecommand{\tightlist}{%
  \setlength{\itemsep}{0pt}\setlength{\parskip}{0pt}}
\setcounter{secnumdepth}{-\maxdimen} % remove section numbering
\ifLuaTeX
  \usepackage{selnolig}  % disable illegal ligatures
\fi
\IfFileExists{bookmark.sty}{\usepackage{bookmark}}{\usepackage{hyperref}}
\IfFileExists{xurl.sty}{\usepackage{xurl}}{} % add URL line breaks if available
\urlstyle{same} % disable monospaced font for URLs
\hypersetup{
  pdftitle={STK-IN4300},
  pdfauthor={Kyunhee Park},
  hidelinks,
  pdfcreator={LaTeX via pandoc}}

\title{STK-IN4300}
\author{Kyunhee Park}
\date{2024-09-11}

\begin{document}
\maketitle

\hypertarget{summary-statistics-table}{%
\section{1. Summary Statistics Table}\label{summary-statistics-table}}

Summarizes the information in the data on a variable level.

\hypertarget{dataset-description}{%
\subsection{1.1 Dataset description}\label{dataset-description}}

\begin{itemize}
\tightlist
\item
  Bike rental counts in a city.
\item
  Time series data: 01.12.2017 - 30.11.2018.
\item
  A day is divided to one-hour length time slot.
\item
  Possible use case: Prediction of rental bike demands, on certain time
  or seasons.
\end{itemize}

\hypertarget{target-variable}{%
\subsubsection{1.1.1 Target variable}\label{target-variable}}

\begin{itemize}
\tightlist
\item
  Rented.Bike.Count: Target variable
\end{itemize}

\hypertarget{categorical-variables}{%
\subsubsection{1.1.2 Categorical
variables}\label{categorical-variables}}

\begin{itemize}
\tightlist
\item
  Date: Date in sequence, 01.12.2017 - 30.11.2018.
\item
  Hour: 0-23 (12AM - 11PM). Reasonable to treat this variable as a
  categorical variable.
\item
  Seasons: Autumn, Winter, Spring, Summer -\textgreater{} Transform to a
  factor variable, by applying as.factor()
\item
  Holiday: No Holiday, Holiday -\textgreater{} Transform to a factor
  variable, by applying as.factor()
\item
  Functioning.Day: Yes, No -\textgreater{} ``No'' rows will be filtered
  out, as no additional information is obtainable.
\end{itemize}

\hypertarget{continous-variables}{%
\subsubsection{1.1.3 Continous variables}\label{continous-variables}}

\begin{itemize}
\tightlist
\item
  Temperature
\item
  Humidity
\item
  Wind.speed
\item
  Visibility
\item
  Dew.point.temperature
\item
  Solar.Radiation
\item
  Rainfall
\item
  Snowfall
\end{itemize}

\hypertarget{summary-statistics-table.}{%
\subsection{1.2 Summary Statistics
Table.}\label{summary-statistics-table.}}

The table below shows the summary table of preprocessed dataset.

\begin{verbatim}
##          Date      Rented.Bike.Count      Hour      Temperature.C.  
##  01/01/2018:  24   Min.   :   2.0    7      : 353   Min.   :-17.80  
##  01/02/2018:  24   1st Qu.: 214.0    8      : 353   1st Qu.:  3.00  
##  01/03/2018:  24   Median : 542.0    9      : 353   Median : 13.50  
##  01/04/2018:  24   Mean   : 729.2    10     : 353   Mean   : 12.77  
##  01/05/2018:  24   3rd Qu.:1084.0    11     : 353   3rd Qu.: 22.70  
##  01/06/2018:  24   Max.   :3556.0    12     : 353   Max.   : 39.40  
##  (Other)   :8321                     (Other):6347                   
##   Humidity...    Wind.speed..m.s. Visibility..10m. Dew.point.temperature.C.
##  Min.   : 0.00   Min.   :0.000    Min.   :  27     Min.   :-30.600         
##  1st Qu.:42.00   1st Qu.:0.900    1st Qu.: 935     1st Qu.: -5.100         
##  Median :57.00   Median :1.500    Median :1690     Median :  4.700         
##  Mean   :58.15   Mean   :1.726    Mean   :1434     Mean   :  3.945         
##  3rd Qu.:74.00   3rd Qu.:2.300    3rd Qu.:2000     3rd Qu.: 15.200         
##  Max.   :98.00   Max.   :7.400    Max.   :2000     Max.   : 27.200         
##                                                                            
##  Solar.Radiation..MJ.m2.  Rainfall.mm.     Snowfall..cm.       Seasons    
##  Min.   :0.0000          Min.   : 0.0000   Min.   :0.00000   Autumn:1937  
##  1st Qu.:0.0000          1st Qu.: 0.0000   1st Qu.:0.00000   Spring:2160  
##  Median :0.0100          Median : 0.0000   Median :0.00000   Summer:2208  
##  Mean   :0.5679          Mean   : 0.1491   Mean   :0.07769   Winter:2160  
##  3rd Qu.:0.9300          3rd Qu.: 0.0000   3rd Qu.:0.00000                
##  Max.   :3.5200          Max.   :35.0000   Max.   :8.80000                
##                                                                           
##        Holiday    
##  Holiday   : 408  
##  No Holiday:8057  
##                   
##                   
##                   
##                   
## 
\end{verbatim}

\hypertarget{bad-data-visualization}{%
\section{2. Bad Data Visualization}\label{bad-data-visualization}}

For at least one categorical and one continuous variable in your data,
make a bad plot and explain why it is bad and possibly even misleading.

\hypertarget{bad-data-visualization---categorical-variable}{%
\subsection{2.1 Bad Data Visualization - Categorical
variable}\label{bad-data-visualization---categorical-variable}}

\begin{itemize}
\tightlist
\item
  X axis: Holiday/No Holiday
\item
  Y axis: Average Rented.Bike.Count
\end{itemize}

\includegraphics{report_files/figure-latex/2.1-categorical-1.pdf} - Does
not provide information regarding seasonal effect. - Lack of information
regarding distribution - Quantile, Median, Min, Max

\hypertarget{bad-data-visualization---continous-variable}{%
\subsection{2.2 Bad Data Visualization - Continous
variable}\label{bad-data-visualization---continous-variable}}

\begin{itemize}
\tightlist
\item
  X axis: Temperature
\item
  Y axis: Rented.Bike.Count
\end{itemize}

\includegraphics{report_files/figure-latex/2.2-continous-1.pdf} - Does
not include seasonal effect. - Rented.Bike.Count increases as the
temperature increases, but no additional information is obtainable.

\hypertarget{good-data-visualization}{%
\section{3. Good Data Visualization}\label{good-data-visualization}}

Provide a good version of the bad plot(s) from Problem 2 and explain how
the plot has been improved.

\hypertarget{good-data-visualization---categorical-variable}{%
\subsection{3.1 Good Data Visualization - Categorical
variable}\label{good-data-visualization---categorical-variable}}

\hypertarget{improvements}{%
\subsubsection{3.1.1 Improvements}\label{improvements}}

\begin{itemize}
\tightlist
\item
  Include seasonal effect.
\item
  Provide the distribution of the bike rentals per Seasons.
\item
  Provide the distribution of the bike retnals per Hour.
\end{itemize}

\begin{verbatim}
## Warning: package 'gridExtra' was built under R version 4.3.3
\end{verbatim}

\includegraphics{report_files/figure-latex/3.1-categorical-1.pdf} \#\#\#
3.1.2 Interpretation - No Holiday has overall higher demands. - The
ranges of 25th quantile-75th quantile are similar in both on Holiday/No
Holiday. - More frequent extreme values in No Holiday, likely to be
during the rush-hour. - The demand for bike rental is highest during the
rush-hour, Hour variable 8 and 18.

\hypertarget{good-data-visualization---continous-variable}{%
\subsection{3.2 Good Data Visualization - Continous
variable}\label{good-data-visualization---continous-variable}}

\hypertarget{improvements-1}{%
\subsubsection{3.1.1 Improvements}\label{improvements-1}}

\begin{itemize}
\tightlist
\item
  Include the seasonal effect.
\item
  Provide the distribution of Rented.Bike.Count.
  \includegraphics{report_files/figure-latex/3.2-continous-1.pdf} \#\#\#
  3.2.1 Interpretations
\item
  The range of temperature is similar in the Autumn and Spring.
\item
  In summer, there are more observations above 2500 Compare to Autumn
  and Spring.
\item
  In winter, there aren't observations over 1000.
\end{itemize}

\hypertarget{simple-analysis}{%
\section{4. Simple analysis}\label{simple-analysis}}

Perform linear regression on the data, including some method for
performing model selection. Include some measure of performance on the
final model. Based both on the analysis and your previous
visualizations, evaluate whether a linear model is sufficient.

\hypertarget{data-preprocessing}{%
\subsection{4.1 Data preprocessing}\label{data-preprocessing}}

\hypertarget{categorical-variable.}{%
\subsubsection{4.1.1 Categorical
variable.}\label{categorical-variable.}}

\hypertarget{continous-variable.}{%
\subsubsection{4.2.1 Continous variable.}\label{continous-variable.}}

\hypertarget{feature-selection.}{%
\subsection{4.2 Feature selection.}\label{feature-selection.}}

\hypertarget{correlation-plot}{%
\subsubsection{4.2.1 Correlation plot}\label{correlation-plot}}

\hypertarget{model-selection}{%
\subsection{4.3 Model selection}\label{model-selection}}

Selected Method: Multiple Linear Regression.

\hypertarget{validation}{%
\subsection{4.4 Validation}\label{validation}}

K-fold Cross Validation.

\hypertarget{model-performance.}{%
\subsection{4.5 Model Performance.}\label{model-performance.}}

\hypertarget{analysis-assessment}{%
\section{5. Analysis assessment}\label{analysis-assessment}}

\hypertarget{time-series-data-seasonal-effect-and-date-column-are-not-considered.}{%
\subsection{5.1 Time series data, seasonal effect and Date column are
not
considered.}\label{time-series-data-seasonal-effect-and-date-column-are-not-considered.}}

\hypertarget{hour-variable-is-important---obvious-that-it-moves-based-on-the-time.-peaks-at-rush-hour.}{%
\subsection{5.2 Hour variable is important - obvious that it moves based
on the time. Peaks at rush
hour.}\label{hour-variable-is-important---obvious-that-it-moves-based-on-the-time.-peaks-at-rush-hour.}}

\begin{verbatim}
## `summarise()` has grouped output by 'Hour'. You can override using the
## `.groups` argument.
\end{verbatim}

\begin{verbatim}
## Warning: Using `size` aesthetic for lines was deprecated in ggplot2 3.4.0.
## i Please use `linewidth` instead.
## This warning is displayed once every 8 hours.
## Call `lifecycle::last_lifecycle_warnings()` to see where this warning was
## generated.
\end{verbatim}

\includegraphics{report_files/figure-latex/5.2-1.pdf} \#\# 5.3
Generalized Linear Model.

\hypertarget{interaction-term.}{%
\subsection{5.4 Interaction term.}\label{interaction-term.}}

\end{document}
